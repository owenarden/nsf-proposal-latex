\section*{Facilities, Equipment, and Other Resources}

\subsection*{Overview and Environment} Started in 1997, the Baskin School of Engineering comprises seven departments: Applied Mathematics and Statistics, Biomolecular Engineering, Computational Media, Computer Science, Computer Engineering, Electrical Engineering, and Technology Management. Faculty in the seven departments collaborate across disciplinary lines on several research and education endeavors.

\subsection*{Space} The Baskin School of Engineering occupies principally the Jack Baskin Engineering and Engineering 2 buildings, comprising a total of 192,500 assignable square feet. Some laboratories and offices are also in the new Biomedical Sciences and Engineering Building (completed in 2012), the Physical and Biological Sciences Building, and the Sinsheimer Laboratory Building – all of which are located in the Science Hill complex of campus (area encircled in red). This facilitates collaboration with our colleagues in the physical and biological sciences, and ensures that faculty have access to the laboratory, office and classroom space required. On the west side of Santa Cruz, outside the main campus, BSOE has a set of advanced material sciences laboratories at 2300 Delaware Ave. (formally a Texas Instruments semiconductor fabrication plant). Several BSOE faculty also work closely with colleagues at the adjacent UCSC Long Marine Laboratory and Marine Sciences Campus.

\subsection*{Classroom/Auditorium Space} On the first floor of the Engineering 2 building, just off the courtyard, is the “Simularium”, a 100-seat, 2,000-square foot classroom/lecture space equipped with computers, presentation equipment, and video capabilities. This classroom is often used for workshops, summer courses, short courses and lectures. Across the courtyard is the Jack Baskin Engineering Auditorium, a 2,600 square foot classroom/lecture hall constructed in 2004. The JBE Auditorium is equipped with state of the art video equipment and individual seating stations.

\subsection{Computing} To support undergraduate education, research and graduate instruction, the Baskin School of Engineering maintains and operates several computer laboratories with a variety of workstations and servers, as well as a massively parallel machine. The ITS/BSOE computing support team operates the campus network, which interconnects computers, workstations, instructional computing labs, and computer-equipped classrooms with each other and the Internet. In addition, wireless access is available across campus. We maintain a high-speed 100/1000 megabit-per- second network with 1/10 gigabit-per-second fiber optic backbones and redundant core routers and paths. Most areas of BSOE buildings are covered by wireless networking of various types (802.11g/n). The BSOE computing network has redundant connections to the main campus network.

BSOE supports the following:
\begin{itemize}
    \item Central fileservers for core services such as mail, name service, file sharing, and backup
    \item Several general-access Unix systems
    \item Multiple compute servers
    \item Research computing clusters
    \item Several general-use research computing clusters, in addition to the clusters used by individual
research groups. These clusters are available to all faculty and graduate students for general-
purpose computations.
    \item The most relevant computer-aided-design software packages, including Agilent Advanced
Design System (ADS), Cadence IC design tools, Maple, Matlab, Mentor Graphics, National
Instruments Labview, also available are Altera, Synopsys, and Xilinx design software.
    \item The Baskin Engineering building has full machine and electronics shops.
\end{itemize}

In addition to these facilities, the Baskin School of Engineering maintains and operates several
specialized research laboratories.

UC Santa Cruz is within commuting distance of several universities and industrial laboratories,
including Stanford University, UC Berkeley, Google, HP Labs, IBM Almaden Research Center,
Lawrence Livermore, Microsoft, NASA, NEC Labs, Sandia and SRI International.
